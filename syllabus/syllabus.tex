\documentclass[12pt]{article}

\usepackage[margin=1.0in]{geometry}
\input{../../syllabi/preamble}


%%%unchanged
\newcommand{\coursewebpageurl}{https://github.com/kapelner/QC_\coursedept_\coursenumber_\semester_\the\year}
\newcommand{\coursewebpagelink}{\href{\coursewebpageurl}{course homepage}}
\newcommand{\slackurl}{https://QC\coursedept\coursenumber\semester\the\year.slack.com/}
\newcommand{\slacklink}{\href{\slackurl}{slack}}

%%edit these!!
\newcommand{\coursedept}{Math}
\newcommand{\coursenumber}{343}
\newcommand{\sixhundredsection}{643}
\newcommand{\coursenumbercrosslisted}{/ 643}
\newcommand{\professorname}{Adam Kapelner, PhD.}
\newcommand{\professorcontactinfo}{@kapelner on \slacklink~in a public channel}
\newcommand{\professoroffice}{604 Kiely Hall}
\newcommand{\semester}{Spring}
\newcommand{\numcredits}{3}
\newcommand{\lectimeandloc}{Tues and Thurs 5:20 -- 6:35PM / KY 258}
\newcommand{\requiredlabtimeandloc}{}
\newcommand{\tataofficehourtimeandloc}{see \coursewebpagelink}
\newcommand{\numtheoryhws}{6--9}
\newcommand{\lastdatetimetohandinhomeworks}{Dec 15 at noon}
\newcommand{\midtermonedateandlocation}{Thursday, October 7 on zoom during class time}
\newcommand{\midtermtwodateandlocation}{Thursday, November 11 on zoom during class time}
\newcommand{\finaldateandlocation}{TBD but on zoom}


\input{../../syllabi/_header}



\section*{Course Overview}

MATH 343 / 643. Computational Statistics for Data Science. 3 hr.; 3 cr. Prereq.: MATH 341 or 641. Coreq.: MATH 342W or 642. Mixture models, EM algorithm, Metropolis-within-Gibbs sampling, permutation tests, the bootstrap, the Kaplan-Meier estimator, the Cox model, T and F tests for the linear model, Gauss-Markov theorem, Bayesian linear regression: Ridge and Lasso. Causality and the randomized experiment, randomization tests. Focus on computation. Special topics. Students cannot receive credit for both: MATH 343 and 643. Fall, Spring \\

\section*{The Four Data Science Core Classes}

This course is the last and most advanced of the four data science core classes as it builds on the previous three. In Math 340 you learn probability tools and a repertoire of random variable models. In Math 341, you learned the foundations of statistics from both the Frequentist and Bayesian perspectives. Here, we continue with more advanced Frequentist and Bayesian methods. The class will be mostly computational as you now are learning / learned computational skills in Math 342W. Thus, \textbf{this is not your typical mathematics course.}  This course will do lots of modeling of real-world situations using data via the \texttt{R} statistical language.

We will also be able to delve more deeply into the concepts of Math 342W including linear regression from a statistical point of view deriving all the usual linear regression techniques you may have seen in Econometrics. Further, we will also delve into the abyss of modeling and ask questions about correlation vs causation and develop causal models from the ground up. We will discuss the theory of randomization, clinical trials and A/B testing.

If we have time, we will get to advanced machine learning methods such as clustering, reinforcement learning and deep learning.

\input{../../syllabi/_DSS_core}

Examining the above, we note that MATH 341 and 343 form a series of two statistics courses: the first, theoretical with traditional topics and the second, computational with modern topics. Both heavily rely on the theoretical topics taught in MATH 340.

%\pagebreak
%\section*{Tentative Day-by-Day Schedule}
%
%Lectures and their topics with rough time estimates per topic are below:
%
%\begin{enumerate}
\item[Day 1] [35min] the two-proportion z-test as a Wald test, the pooled proportion estimator;  [10min] the approximate two-sample t-test as an approximate two-sample z-test (Wald test); 
[15min] CI for the difference of two proportions

\item[Day 2] [35min] Grid sampling, distribution sampling via kernel grid sampling, disadvantages of grid sampling; [40min] systematic sweep Gibbs sampling, burning the chain, sampling from the semi-conjugate NIG mode

\item[Day 3] [20min] Autocorrelation grid sampling, thinning the chain; [30min] approximate inference with Gibbs samples; [25min] change point detection model  

\item[Day 4] [55min] normal mixture model with data augmentation; [20min] Bayes Factors part I

\item[Day 5] [15min] Bayes Factors part II; [60min] Metropolis algorithm, Metropolis-Hastings algorithm, metropolis-within-Gibbs, transition kernels

\item[Day 6] [10min] equivalence of the two-sided z and $\chi^2$ tests, equivalence of the two-sided t and F tests; shape of the $\chi^2$ and F distributions;  [30min] $\chi^2$ goodness of fit test for multinomial parameters, its $H_0$ and $H_a$, observed and expected counts, the test statistic; [35min] the $\chi^2$ test of independence among two categorical variables, its $H_0$ and $H_a$, observed and expected counts, the test statistic

\item[Day 7] [10min] review of Type I/II errors; [40min] the multiple hypothesis testing / comparison problem, the $2 \times 2$ frequency table of test results, false discoveries; [10min] definition of familywise error rate (FWER), weak FWER control; [10min] Bonferroni procedure 

\item[Day 8] [15min] Sidak procedure; [25min] Simes procedure; [15min] definition of false discovery proportion, false discovery rate (FDR), setting for equivalence of FWER and FDR, statement that the Simes procedure provides strong control of the FDR; [20min] proof that p-values are uniformly distributed under $H_0$

\item[Day 9] \inblue{Midterm I Review}
\item[Day 10] \inred{Midterm I}

%85min
\item[Day 11] [50min] derivation of the score test as a Wald test, the score test statistic for the logistic distribution with known scale parameter; [15min] definition of likelihood ratio (LR) test statistic, statement of its asymptotic convergence to a $\chi^2_1$, LR test (LRT) derivation of the LR statistic for the iid Bernoulli DGP; [10min] Proof of the asymptotic convergence of the LR test statistic part I

%65min
\item[Day 12] [15min] Proof of the asymptotic convergence of the LR test statistic part II;  [50min] generalized likelihood ratio test for reduced models nested in a full model, statement of asymptotic convergence to a $\chi^2$ with degrees of freedom equal to the difference of number of parameters, demonstration that it differs from the goodness of fit test, demonstration in the normal iid DGP; [10min] visualizing the Wald, score and LR tests

%85min
\item[Day 13] [5min] testing entire DGPs; [10min] definition and illustration of the empirical CDF; [20min] one-sample Kolmogorov-Smirnov (KS) test, statement of Kolmogorov distribution and its critical values; [10min] tests of DGP equivalence among two populations; [15min] two-sample KS test; [15min] intro to the the two-sample permutation test



%65min
\item[Day 14] [45min] the two-sample permutation test partitioning the master population, dsicussion of possible test statistics, computational construction of its RET via resampling; [30min] the nonparametric bootstrap procedure, typical use cases, approximate CI construction, approximate hypothesis testing

\item[Day 15] [30min] Identifiability [30min] statistical sufficiency; [15min] statistical ancillarity

\item[Day 16] [34min] exponential families: sufficiency and ancillarity terms; [20min] Normal posterior under laplace prior = Lasso; [20min] Normal posterior under normal prior = Ridge

%80min
\item[Day 17] [20min] causal inference vs statistical inference, counfounding variable, an illustration of one scenario with confounding and one scenario without confounding; [15min] definition of a two-arm treatment vs control experiment, definition of assignment / allocation / manipulation; [10min] the Rubin causal model, counterfactuals, additive treatment effect, selection bias inducing bias in the naive treatment estimator; [30min] response model linear in confounder and noise, noise as an approximate normal realization, the explicit bias term and its explanation;

%70min
\item[Day 18] [30min] randomized experiments, bernoulli trial design, completely randomized trial design, statment that causal estimates are unbiased over errors and randomized assignments, statement that bias is small in any random assignment; [45min] the randomization test vs. the permutation test, hypothesis testing and confidence intervals

\item[Day 19] \inblue{Midterm II Review}
\item[Day 20] \inred{Midterm II}

\item[Day 21] [30min] optimal experimental design; [30min] restricted randomization strategies such as rerandomization, matching; [15min] sequential experimental design

\item[Day 22] [35min] Weibull modeling of survival; [35min] Kaplan-Meier estimate

\item[Day 23] [30min] Proportional Cox model; [30min] Poisson Regression]; [15min] Zero-Inflated Poisson regression

\item[Day 24] [30min] Negative Binomial regression; [20min] Beta regression; [15min] Review of multivariate normal rv

%70min
\item[Day 25] [55min] Derivation of T-test and F-test and chi-squared test linear regression estimator; [20min] derivation of variance-covariance matrix of the least squared estimators

%75min
\item[Day 26] [30min] Introduction to neural networks; [45min] estimation of parameters in neural networks using optimization algorithms

%80min
\item[Day 27] [30min] Introduction to deepl learning; [45min] convolutional neural networks for image modeling

\item[Day 28] \inblue{Final Review}

\end{enumerate}





\subsection*{Prerequisites}

MATH 341 / 641 or a foundations of Frequentist and Bayesian statistics course. Critical is coverage of hypothesis testing, confidence intervals, credible regions, maximum likelihood. Implicitly, MATH 340 / 640 is also a prerequisite. At times, we will be drawing on these concepts as well so keep those notes handy.

\subsection*{Corequisites}

MATH 342W / 642 or a foundations of data science course. Critical is coverage of supervised learning, basic computing, linear regression and logistic regression.

\section*{Course Materials}

\paragraph{Textbook:} I will be referencing Larry Wasserman's \emph{All of Statistics: A concise course in statistical inference} which can be purchased on \href{https://www.amazon.com/dp/0387402721}{Amazon} and Casella and Berger's \emph{Statistical Inference} which can be purchased on \href{https://www.amazon.com/dp/8131503941}{Amazon}. There is no excuse not to have these books. They are \textit{required}. However, I will not ususally be teaching \qu{from the book} --- most of the material in the class comes from the lecture notes. The textbooks are a way to get ``another take'' on the material and they will only cover about only half of the material done in class. For the other half, you will have to make use of other resources. I also recommend Rice's \emph{Mathematical Statistics and Data Analysis}, 3rd edition which can be purchased on \href{https://www.amazon.com/dp/0534399428}{Amazon} as well but I will not reference it during class.

\paragraph{Computer Software:} During lectures, there will be demos using \texttt{R} which is a free, open source statistical programming language and console. You can download it from: \url{http://cran.mirrors.hoobly.com/}.As this course is coreq'd with MATH 342W, this course has a lot of programming in \texttt{R} for the homeworks.

\input{../../syllabi/_calculator}

%\input{../../syllabi/_the650section}

\input{../../syllabi/_use_of_slack}

\input{../../syllabi/_announcements_on_slack}

\input{../../syllabi/_standard_class_meetings}

%\input{../../syllabi/_jewish_holiday_reschedule}

%\input{../../syllabi/_zoom_policies}

%\input{../../syllabi/_lecture_upload}

\section*{Homework}

There will be \numtheoryhws~theory homework assignments and \numtheoryhws~practice homework assignments (labs). Homeworks will be assigned and placed on the \coursewebpagelink~ and will usually be due a week later in class. Homework will be \textbf{graded} out of 100 with extra credit getting scores possibly $> 100$. I will be doing the grading and will grade an \textit{arbitrary subset of the assignment} which is determined after the homework is handed in. 

\input{../../syllabi/_theory_hws_submission_text}
\input{../../syllabi/_philosophy_hws}

\input{../../syllabi/_time_spent_hws_3_cr}

\input{../../syllabi/_late_hw_policy}

\input{../../syllabi/_latex_hw_bonus_policy}

\input{../../syllabi/_hw_ec_policy}

\section*{Examinations}

\input{../../syllabi/_examination_text}

\input{../../syllabi/_standard_exam_schedule}

\subsection*{Exam Policies and Materials}

\input{../../syllabi/_examination_policies}

%\input{../../syllabi/_zoom_examination_policies}

\input{../../syllabi/_standard_cheat_sheet_policy}


\input{../../syllabi/_cheating_on_exams_and_missing_exams}
\input{../../syllabi/_special_services}

\input{../../syllabi/_class_participation}

%\input{../../syllabi/_zoom_attendance}

\input{../../syllabi/_343_grading_and_grading_policy}

\subsection*{The Grade Distribution}

As this is a small and advanced class, the class curve will be quite generous. I'm expecting approximately 40\% A's and 40\% B's. If you do your homework and demonstrate understanding on the exams, you should expect to be rewarded with an A or a B. C's are for those who \qu{dropped out} somewhere mid-semester or who cannot demonstrate basic understanding.

\input{../../syllabi/_grade_checking_on_gradesly}

\input{../../syllabi/_auditing_policy}

\end{document}